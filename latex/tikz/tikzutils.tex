%% https://tex.stackexchange.com/questions/31336/how-can-i-display-an-array-as-in-the-data-structure-from-computer-science-not

\usetikzlibrary{calc}

\newcounter{nodeid}
\newcounter{nodex}
\newcounter{nodeheight}

\tikzstyle{box} = [minimum size=6mm, draw, rectangle]

\newcommand{\nodes}[1]{%
	\foreach \num in {#1}{
		% \node[minimum size=6mm, draw, rectangle] (\arabic{nodeid}) at (\arabic{nodex}, 0) {\num};
		\node (\arabic{nodeid}) [box] at (\arabic{nodex}, -\nodeheight) {{\arabic{\num}};
		\stepcounter{nodex}
		\stepcounter{nodeid}
	}
}

\newcommand{\newblock}{
	\stepcounter{nodeheight}
}

\newcommand{\bracket}[4][]{% [text] {from, to, lvl}
	\draw (#2.south west) ++($(-.1, -.1) + (-.1*#4, 0)$)
	-- ++($(0,-.1) + (0,-#4*1.25em) $)
	-- node [below] {#1} ($(#3.south east) + (.1,-.1) + (.1*#4, 0) + (0,-.1) + (0,-#4*1.25em)$)
	-- ++($(0,#4*1.25em) + (0,.1)$);%
}

\newcommand{\setopacity}[1]{
	\pgfsetstrokeopacity{#1}
	\pgfsetfillopacity{#1}
}

\makeatletter
\let\@tikzpicture\tikzpicture
\def\tikzpicture{\@tikzpicture \setcounter{nodex}{1} \setcounter{nodeheight}{1}}
\makeatother