\chapter{Què és ordenar un vector?}
Un vector és un seguit d'elements en un ordre qualsevol.
Quan n'ordenem un estem canviant de posició els elements que conté per tal que estiguin ordenats d'una manera determinada.

En aquest treball els elements a ordenar són números, que ordenarem de forma ascendent.

\section{Què podem ordenar?}
No podem ordenar elements sense una referència que ens indiqui quin va primer.
Això ho veiem en números i lletres: primer va l'1 i el següeix el 2, comença la A i a continuació la B.

Si volem ordenar hortalisses no existeix cap sistema com amb els números o les lletres i se'ns plantegen dues opcions:
\begin{enumerate}[label={\alph*)}]
	\item inventar-nos un sistema de referència: primer les cols, després les pastanagues, i finalment els espinacs o
	\item buscar una propietat comuna en totes les nostres verdures i ordenar-les en funció d'aquesta (alfabèticament pel nom, de menor a major pes, etc.)
\end{enumerate}

Ordenem el que ordenem el procediment és el mateix, i els mètodes aqui utilitzats per ordenar números també són vàlids per ordenar paraules o hortalisses.

\pagebreak
\section{Per què ordenem?}
Ordenem números pel mateix motiu que ordenem l'armari, l'habitació o la casa sencera: el temps que dediquem ara a ordenar el recuperem a l'hora de buscar.

Si guardem totes les garanties dels electrodomèstics en una carpeta no patirem (tant) quan se n'espatlli un.

L'índex d'aquest document permet navegar-lo amb facilitat, però no faria servei si les pàgines no estiguéssin ordenades. Si aquesta pàgina, la {\thepage} es trobés entre la {\the\numexpr \thepage + 6} i la {\the\numexpr \thepage - 2} la numeració dels fulls faria més nosa que servei.

De la mateixa manera agraïm que els diccionaris (en paper) ordenin les paraules alfabèticament, si ho féssin aleatòriament vendrien pocs exemplars.