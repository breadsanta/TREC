\chapter{Introducció}
Quan algú em pregunta de què va el meu TREC i responc amb "Algoritmes d'ordenació" veig davant meu cares d'indiferència o, en el millor dels casos, confusió.
No els culpo: quan el Xavi, el meu tutor, em va proposar el tema vaig reaccionar de la mateixa manera.
Això va canviar quan vaig començar a indagar, sentia com es despertava dins meu una curiositat cada vegada més intensa. 
No cal dir que vaig acabar acceptant la seva proposta.

L'objectiu que em plantejo en aquest treball és entendre per tal de poder analitzar i, en última instància, comparar el rendiment de diversos algoritmes d'ordenació.

El document s'estructura en tres grans blocs: dos primers d'algoritmes, uns recursius i uns altres iteratius, on cada algoritme té el seu apartat amb tres seccions: explicació del funcionament, implementació i finalment els temps d'execució. Tots aquests temps d'execució tornen a aparèixer en tercer bloc, on s'analitzen i es comparen.

Els algortimes iteratius analitzats són el \textit{selection sort}, l'\textit{insertion sort}, el \textit{bubble sort} i el \textit{shell sort}. De recursius n'hi ha un parell: \textit{quick sort} i \textit{merge sort}. N'hi ha molts més, tant de recursius com d'iteratius, però tard o d'hora s'ha de posar un límit. Els algoritmes escollits es troben entre els més coneguts i mostren la diversitat del camp, diversitat tant en la manera d'enfocar el problema com en els resultats obtinguts, però la tria no deixa de ser arbitrària.